\documentclass[12pt]{article}
\usepackage[utf8]{inputenc}
\usepackage{amsmath}
\usepackage{fancyhdr}

\title{M2AA3 Project 2}
\author{Miki Ivanovic}
\date{March 2019}

\pagestyle{fancy}

\setlength{\headheight}{15pt}
\lhead{Miki Ivanovic}
\chead{M2AA3 Project 2}
\rhead{01345671}

\begin{document}

\maketitle

\section{}

We use the definition of the inner product on $V$
$$\left\langle
\frac{df}{dx}
,
\sin{\left(\frac{m \pi \cdot}{d}\right)}
\right\rangle
=
\int_{-d}^{d}
\frac{df}{dx}
\sin\left(\frac{m \pi x}{d}\right)
dx$$
We integrate by parts, setting $u' = \frac{df}{dx}$ and $v = \sin\left(\frac{m\pi x}{d}\right)$ to obtain
$$
\left[
f(x)
\sin{\left(
\frac{m \pi x}{d}
\right)}
\right]_{-d}^{d}
-
\int_{-d}^{d}
\frac{m\pi}{d}
f(x)
\cos\left(\frac{m\pi x}{d}\right)
dx
$$
which is valid for $m \ne 0$
\\The left limit evaluates to zero since the inner terms of $\sin$
are integer multiples of $\pi$.
\\We are left with the integrand on the right which is 
$$
-\frac{m\pi}{d}
\int_{-d}^{d}f(x)\cos\left(\frac{m\pi x}{d}\right)dx
=
-\frac{m\pi}{d}
\left\langle
f(x),\cos\left(\frac{m\pi x}{d}\right)
\right\rangle
$$
for $m = 1 \rightarrow n$ this gives the result $-m\pi a_m$.
\\
The second inner product follows simillarly:
$$\left\langle
\frac{df}{dx}
,
\cos{\left(\frac{m \pi \cdot}{d}\right)}
\right\rangle
=
\int_{-d}^{d}
\frac{df}{dx}
\cos\left(\frac{m \pi x}{d}\right)
dx$$
We integrate by parts, setting $u' = \frac{df}{dx}$ and $v = \cos\left(\frac{m\pi x}{d}\right)$ to obtain
$$
\left[
f(x)
\cos{\left(
\frac{m \pi x}{d}
\right)}
\right]_{-d}^{d}
+
\int_{-d}^{d}
\frac{m\pi}{d}
f(x)
\sin\left(\frac{m\pi x}{d}\right)
dx
$$
The left limit evaluates to zero since $f(d)=f(-d)$, and $\cos$ is even about zero.
\\We are left with the integrand on the right which is 
$$
\frac{m\pi}{d}
\int_{-d}^{d}f(x)\sin\left(\frac{m\pi x}{d}\right)dx
=
\frac{m\pi}{d}
\left\langle
f(x),\sin\left(\frac{m\pi x}{d}\right)
\right\rangle
$$
which is $m\pi b_m$ for $m=1 \rightarrow n$.

\section{}

Let $\bar{f}_{n}^{*}$ be the best approximation to $\frac{df}{dx}$ in $\|\cdot\|$ from $V_n$.
Then we have
$$
\bar{f}_{n}^{*} = 
\frac{\bar{a}_0}{2} + 
\sum_{m=1}^{n}
\left[
\bar{a}_m\cos\left(\frac{m\pi x}{d}\right) +  \bar{b}_m\sin\left(\frac{m\pi x}{d}\right)
\right]
$$
where
$$
\bar{a}_m = 
\frac{1}{d}
\left\langle
\frac{df}{dx},
\cos\left(\frac{m\pi\cdot}{d}\right)
\right\rangle
, m = 0 \rightarrow n
$$
$$
\bar{b}_m = 
\frac{1}{d}
\left\langle
\frac{df}{dx},
\sin\left(\frac{m\pi\cdot}{d}\right)
\right\rangle
, m = 1 \rightarrow n
$$
Note that for $m=1 \rightarrow n$, $\bar{a}_m = \frac{m\pi b_m}{d}$ and  $\bar{b}_m = -\frac{m\pi a_m}{d}$
\\
Moreover, 
$$
\bar{a}_0=
\frac{1}{d}\left\langle\frac{df}{dx},1\right\rangle =
\frac{1}{d}\int_{-d}^{d}\frac{df}{dx}dx=
\frac{1}{d}\left[f(x)\right]_{-d}^{d}=0
$$
\\
Now consider
\begin{align*}
\frac{df_{n}^{*}}{dx}
&=
\frac{d}{dx}\left(
\frac{a_0}{2} +
\sum_{m=1}^{n}
\left[
a_m\cos\left(\frac{m\pi x}{d}\right) +
b_m\sin\left(\frac{m\pi x}{d}\right)
\right]
\right)
\\&=
\sum_{m=1}^{n}
\left[
-\frac{m\pi a_m}{d}\sin\left(\frac{m\pi x}{d}\right) +
\frac{m\pi b_m}{d}\cos\left(\frac{m\pi x}{d}\right)
\right]
\\&=
\sum_{m=1}^{n}
\left[
\bar{b}_m\sin\left(\frac{m\pi x}{d}\right) +
\bar{a}_m\cos\left(\frac{m\pi x}{d}\right)
\right]
\\&=
\bar{f}_{n}^{*}
\end{align*}
Therefore
$$
\left\|
\frac{df}{dx} -
\frac{df_{n}^{*}}{dx}
\right\|\leq
\left\|
\frac{df}{dx} -
v_n
\right\|
,\forall v_n\in V_n
$$


\section{}

We have
\begin{align*}
\frac{\pi^{2}}{d}\sum_{m=1}^{n}
m^2\left[a_m^2 + b_m^2\right]
&=
d\sum_{m=1}^{n}
\left[
\left(\frac{m\pi b_m}{d}\right)^2 + 
\left(-\frac{m\pi a_m}{d}\right)^2
\right]
\\&=
d\left(
\frac{\bar{a}_{0}^2}{2} +
\sum_{m=1}^{n}
\left[
\bar{a}_{m}^{2} + \bar{b}_{m}^{2}
\right]
\right)
=
\left\|\bar{f}_{n}^{*}\right\|^2
\\&=
\left\|\frac{df_n^*}{dx}\right\|^2
=
\left\|\frac{df}{dx}\right\|^2 -
\left\|\frac{df}{dx} - \frac{df_n^*}{dx}\right\|^2
\leq
\left\|\frac{df}{dx}\right\|^2
\end{align*}
By Bessel's inequality.

\section{}

Consider 
$$
\frac{\pi^2}{d}\left(n+1\right)^2\left(a_{n+1}^2+b_{n+1}^2\right) =
\left\|\frac{df_{n+1}^*}{dx}\right\|^2 - \left\|\frac{df_n^*}{dx}\right\|^2
$$
Since $\left\{\frac{df_n^*}{dx}\right\}_{n=1}^\infty$ converges to $\frac{df}{dx}$, we have 
$$
\lim_{m \rightarrow \infty}
\left[
m^2\left(a_m^2+b_m^2\right)
\right]
= \lim_{m \rightarrow \infty}m^2a_m^2 + 
\lim_{m \rightarrow \infty}m^2b_m^2
= 0
$$
Moreover since the terms are non-negative, we can say each limit tends to zero, and also take square roots to obtain the result.

\section{}

First we find the coefficients $\left\{a_m\right\}_{m=0}^\infty$:
$$
a_m=\frac{1}{d}\left\langle f(x),\cos\left(\frac{m\pi\cdot}{d}\right)\right\rangle =
\left\langle\frac{1}{2}x^2,\cos\left(m\pi\cdot\right)\right\rangle =
\int_{-1}^1 \frac{1}{2}x^2\cos\left(m\pi x\right)dx
$$
For $m=0$ we have
$$
a_0=\int_{-1}^1\frac{1}{2}x^2dx=2\left[\frac{1}{6}x^3\right]_0^1 = \frac{1}{3}
$$
For all other $m$, we integrate by parts, setting $u' = \cos\left(m\pi x\right)$ and $v=\frac{1}{2}x^2$
$$
a_m=\left[\frac{1}{2}x^2\frac{\sin\left(m\pi x\right)}{m\pi}\right]_{-1}^1 -
\int_{-1}^1 x\frac{\sin\left(m\pi x\right)}{m\pi}dx
$$
Integrate by parts again, with $u' = \sin\left(m\pi x\right)$ and $v = x$
\begin{align*}
a_m&=
2\left[\frac{1}{2}x^2\frac{\sin\left(m\pi x\right)}{m\pi}\right]_0^1 +
\left[x\frac{\cos\left(m\pi x\right)}{m^2\pi^2}\right]_{-1}^1 -
\int_{-1}^1\frac{\cos\left(m\pi x\right)}{m^2\pi^2}dx
\\&=
\frac{\sin\left(m\pi\right)}{m\pi} +
2\left[x\frac{\cos\left(m\pi x\right)}{m^2\pi^2}\right]_0^1 -
\left[\frac{\sin\left(m\pi x\right)}{m^3\pi^3}\right]_{-1}^1
\\&=
\frac{2\cos\left(m\pi\right)}{m^2\pi^2} +
2\left[\frac{\sin\left(m\pi x\right)}{m^3\pi^3}\right]_0^1
\\&=
\frac{2(-1)^m}{m^2\pi^2} +
\frac{\sin\left(m\pi\right)}{m^3\pi^3}
\\&= 
\frac{2(-1)^m}{m^2\pi^2}
\end{align*}
\\
Now for $\left\{b_m\right\}_{m=0}^\infty$:
$$
b_m=\left\langle\frac{1}{2}x^2,\sin\left(m\pi\cdot\right)\right\rangle =
\int_{-1}^1\frac{1}{2}x^2\sin\left(m\pi x\right)dx = 0
$$
since the integrand is even about zero.
\\
Therefore we have 
$$
f_n^*(x)=\frac{1}{6} + \frac{2}{\pi^2}\sum_{m=1}^n\frac{(-1)^m}{m^2}\cos(m\pi x) 
$$
and
$$
\frac{df_n^*}{dx}=-\frac{2}{\pi}\sum_{m=1}^n\frac{(-1)^m}{m}\sin(m\pi x)
$$

\section{}

To find $\|f-f_n^*\|^2$ we must first find the integrals $\int_{-1}^1f^2dx$, $\int_{-1}^1ff_n^*dx$ and $\int_{-1}^1{f_n^*}^2dx$

$$
\int_{-1}^1f^2dx = \int_{-1}^1\frac{1}{4}x^4dx = 2\left[\frac{1}{20}x^5\right]_0^1
= \frac{1}{10}
$$
\begin{align*}
\int_{-1}^1ff_n^*dx &=
\int_{-1}^1\frac{1}{12}x^2 + \frac{2}{\pi^2}\sum_{m=1}^n\frac{(-1)^m}{m^2}\frac{1}{2}x^2\cos(m\pi x)dx
\\&= 
2\left[\frac{1}{36}x^3\right]_0^1 + 
\frac{2}{\pi^2}\sum_{m=1}^n\frac{(-1)^m}{m^2}a_m
\\&=
\frac{1}{18} + \frac{4}{\pi^4}\sum_{m=1}^n\frac{1}{m^4}
\end{align*}
\begin{align*}
\int_{-1}^1{f_n^*}^2dx &=
\int_{-1}^1\frac{1}{36}dx + 
\int_{-1}^1\frac{1}{3\pi^2}\sum_{m=1}^n\frac{(-1)^m}{m^2}\cos(m\pi x)dx
\\&+
\int_{-1}^1\frac{2}{\pi^2}\sum_{m=1}^n\frac{(-1)^m}{m^2}\cos(m\pi x)\cdot
\frac{2}{\pi^2}\sum_{m=1}^n\frac{(-1)^m}{m^2}\cos(m\pi x)dx
\\&=
\frac{1}{18} + \frac{1}{3\pi^2}\sum_{m=1}^n\frac{(-1)^m)}{m^2}\cdot
2\left[\frac{\sin(m\pi x)}{m\pi}\right]_0^1 \\&+
\sum_{m=1}^n\sum_{p=1}^n\frac{4(-1)^{m+p}}{m^2p^2\pi^4}
\int_{-1}^1\cos(m\pi x)\cos(p\pi x)dx
\\&=
\frac{1}{18} + 
\sum_{m=1}^n\sum_{p=1}^n\frac{2(-1)^{m+p}}{m^2p^2\pi^4}\int_{-1}^1
\cos\left((m+p)\pi x\right) + \cos\left((m-p)\pi x\right)dx
\\&=
\frac{1}{18} + 
\sum_{m=1}^n\sum_{p=1}^n\frac{4(-1)^{m+p}}{m^2p^2\pi^4}
\left[
\frac{\sin((m+p)\pi x}{(m+p)\pi} + \frac{\sin((m-p)\pi x}{(m-p)\pi}
\right]_0^1
\\&=\frac{1}{18}
\end{align*}
Now, by definition
\begin{align*}
\|f-f_n^*\|^2=\langle f-f_n^*,f-f_n^*\rangle &=
\int_{-1}^1[f-f_n^*]^2dx = \int_{-1}^1 f^2 - 2ff_n^* + {f_n^*}^2dx \\&=
\frac{1}{10} -
2\left(\frac{1}{18} + \frac{4}{\pi^4}\sum_{m=1}^n\frac{1}{m^4}\right) + 
\frac{1}{18} \\&=
\frac{2}{45} - \frac{8}{\pi^4}\sum_{m=1}^n\frac{1}{m^4}
\end{align*}
To find $\left\|\frac{df}{dx}-\frac{df_n^*}{dx}\right\|^2$
we must find the integrals $\int_{-1}^1(\frac{df}{dx})^2dx$,
$\int_{-1}^1\frac{df}{dx}\frac{df_n^*}{dx}dx$
and $\int_{-1}^1\left(\frac{df_n^*}{dx}\right)^2dx$

$$
\int_{-1}^1\left(\frac{df}{dx}\right)^2dx = \int_{-1}^1x^2dx =
2\left[\frac{1}{3}x^3\right]_0^1=\frac{2}{3}
$$
\begin{align*}
\int_{-1}^1\frac{df}{dx}\frac{df_n^*}{dx}dx &=
\int_{-1}^1 -\frac{2}{\pi}\sum_{m=1}^n\frac{(-1)^m}{m^2}x\sin(m\pi x)dx \\&=
-\frac{2}{\pi}\sum_{m=1}^n\frac{(-1)^m}{m^2}\int_{-1}^1x\sin(m\pi x)dx \\&=
0
\end{align*}
\begin{align*}
\int_{-1}^1\left(\frac{df_n^*}{dx}\right)^2dx &=
\int_{-1}^1\left(-\frac{2}{\pi}
\sum_{m=1}^n\frac{(-1)^m}{m}\sin(m\pi x)\right)^2dx \\&=
\sum_{m=1}^n\sum_{p=1}^n\frac{4(-1)^{m+p}}{mp\pi^2}
\int_{-1}^1\sin(m\pi x)\sin(p\pi x)dx \\&=
0
\end{align*}
We have
\begin{align*}
\left\|\frac{df}{dx}-\frac{df_n^*}{dx}\right\|^2 &=
\int_{-1}^1\left(\frac{df}{dx}\right)^2dx -
2\int_{-1}^1\frac{df}{dx}\frac{df_n^*}{dx}dx +
\int_{-1}^1\left(\frac{df_n^*}{dx}\right)^2dx \\&=
\frac{2}{3}
\end{align*}

\section{}

To show:
$$
\frac{d^2[f-f_n^*]}{dx^2}(x) = 
\frac{(-1^n)\cos\left(\left(n+\frac{1}{2}\right)\pi x\right)}
{\cos\left(\frac{\pi x}{2}\right)}
$$
From our evaluations we obtain:
$$
f-f_n^*=\frac{1}{2}x^2 - \frac{1}{6} - \frac{2}{\pi^2}
\sum_{m=1}^n\frac{(-1)^m}{m^2}\cos(m\pi x)
$$
$$
\frac{df}{dx}-\frac{df_n^*}{dx}=x+\frac{2}{\pi}
\sum_{m=1}^n\frac{(-1)^m}{m}\sin(m\pi x)
$$
\begin{align*}
\implies
\frac{d^2[f-f_n^*]}{dx^2}(x) &= 
1 + \frac{d}{dx}\left(\frac{2}{\pi}
\sum_{m=1}^n\frac{(-1)^m}{m}\sin(m\pi x)\right) \\&=
1 + 2\sum_{m=1}^n(-1)^m\cos(m\pi x)
\end{align*}
Now we must evaluate the RHS of the given formula to check it equals our evaluation of the LHS.
We will use the following trigonometric identity.
$$
\sin(A)\sin(B) = \frac{1}{2}(\cos(A-B) - \cos(A+B))
$$
We also use the cosine half-angle formula.
\begin{align*}
\cos((n+\frac{1}{2})\pi x) &=
\cos(n\pi x)\cos(\frac{\pi x}{2}) -
\sin(n\pi x)\sin(\frac{\pi x}{2}) \\&=
\cos(n\pi x)\cos(\frac{\pi x}{2}) -
\frac{1}{2}(\cos((n-\frac{1}{2})\pi x)-
\cos((n+\frac{1}{2})\pi x))
\end{align*}
$$
\implies
\cos((n+\frac{1}{2})\pi x) =
2\cos(n\pi x)\cos(\frac{\pi x}{2}) -
\cos((n-\frac{1}{2})\pi x)
$$
$$
\implies
\frac{(-1^n)\cos\left(\left(n+\frac{1}{2}\right)\pi x\right)}
{\cos\left(\frac{\pi x}{2}\right)} =
(-1)^n2\cos(n\pi x) +
\frac{(-1^{n-1})\cos\left(\left(n-\frac{1}{2}\right)\pi x\right)}
{\cos\left(\frac{\pi x}{2}\right)}
$$
We have now have a recursive relation that we can unfold to obtain our result.
\begin{align*}
\frac{(-1^n)\cos\left(\left(n+\frac{1}{2}\right)\pi x\right)}
{\cos\left(\frac{\pi x}{2}\right)} &=
2\sum_{m=1}^n(-1)^m\cos(m\pi x) +
\frac{(-1)^0\cos\left(\frac{\pi x}{2}\right)}
{\cos\left(\frac{\pi x}{2}\right)} \\&=
1 + 2\sum_{m=1}^n(-1)^m\cos(m\pi x)
\end{align*}
which holds for $x\in(-1,1)$.

\section{}

In order for $\frac{d^2[f-f_n^*]}{dx^2}(x)=0$, we need to choose $x\in(-1,1)$ such that
\\$\cos((n+\frac{1}{2})\pi x)=0$ and $\cos(\frac{\pi x}{2})\neq0$. 
\\Note the second condition implies that that we must have $|x|<1$.
\\To satisfy these conditions, we can immediately see that we need to choose $x=\frac{y}{2n+1}$,\\ where $y\in(-2n+1, 2n-1)$ is an odd integer.


\end{document}
